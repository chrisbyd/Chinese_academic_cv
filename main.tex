%%%%%%%%%%%%%%%%%%%%%%%%%%%%%%%%%%%%%%%%%
% "ModernCV" CV and Cover Letter
% LaTeX Template
% Version 1.1 (9/12/12)
%
% This template has been downloaded from:
% http://www.LaTeXTemplates.com
%
% Original author:
% Xavier Danaux (xdanaux@gmail.com)
%
% License:
% CC BY-NC-SA 3.0 (http://creativecommons.org/licenses/by-nc-sa/3.0/)
%
% Important note:
% This template requires the moderncv.cls and .sty files to be in the same 
% directory as this .tex file. These files provide the resume style and themes 
% used for structuring the document.
%
%%%%%%%%%%%%%%%%%%%%%%%%%%%%%%%%%%%%%%%%%

%----------------------------------------------------------------------------------------
%	PACKAGES AND OTHER DOCUMENT CONFIGURATIONS
%----------------------------------------------------------------------------------------

\documentclass[11pt,a4paper,sans]{moderncv} % Font sizes: 10, 11, or 12; paper sizes: a4paper, letterpaper, a5paper, legalpaper, executivepaper or landscape; font families: sans or roman
\usepackage{standalone}
\usepackage{zh_CN-Adobefonts_external}
\moderncvstyle{classic} % CV theme - options include: 'casual' (default), 'classic', 'oldstyle' and 'banking'
\moderncvcolor{blue} % CV color - options include: 'blue' (default), 'orange', 'green', 'red', 'purple', 'grey' and 'black'

\usepackage{lipsum} % Used for inserting dummy 'Lorem ipsum' text into the template

\usepackage[scale=0.85]{geometry} % Reduce document margins
%\setlength{\hintscolumnwidth}{3cm} % Uncomment to change the width of the dates column
%\setlength{\makecvtitlenamewidth}{10cm} % For the 'classic' style, uncomment to adjust the width of the space allocated to your name

%\usepackage[utf8]{inputenc}

%\usepackage{booktabs}
\usepackage{fontawesome}
\usepackage{marvosym} % For cool symbols.
%\usepackage{hyperref}



%----------------------------------------------------------------------------------------
%	NAME AND CONTACT INFORMATION SECTION
%----------------------------------------------------------------------------------------

 % Your first name
\familyname{陈} % Your last name
\firstname{勇彪}
% All information in this block is optional, comment out any lines you don't need
\title{个人简历}
\address{电子信息与电气工程学院}{上海交通大学}
\mobile{(+86) 18818209278}


%\fax{(000) 111 1113}
 
%\social{github}{stefano-bragaglia}
\email{chenyongbiao0319@sjtu.edu.cn} 



%\homepage{给i挺好/}{My Webpage}

% social link \faGithub, \faSkype, \faLinkedin,\faStackExchange, and \faStackOverflow
% \extrainfo{
%     \faGithub\href{https://github.com/xyz}{ Github} \quad
%     \faLinkedin\href{https://www.linkedin.com/abc/}{ Linkedin} \quad
%     \faSkype\href{https://skype.com/abc}{Skype}
%     }



%\social[linkedin][www.linkedin.com]{name}
% The first argument is %the url for the clickable link, the second argument is the url displayed in the %template - this allows special characters to be displayed such as the tilde in this %example

\photo[70pt][0.3pt]{bio} % The first bracket is the picture height, the second is %the thickness of the frame around the picture (0pt for no frame)
%\quote{Not Attention, Patience is all we need.}

%----------------------------------------------------------------------------------------

\newcommand{\cvdoublecolumn}[2]{%
  \cvitem[.75em]{}{%
    \begin{minipage}[t]{\listdoubleitemcolumnwidth}#1\end{minipage}%
    \hfill%
    \begin{minipage}[t]{\listdoubleitemcolumnwidth}#2\end{minipage}%
    }%
}



\usepackage{multibbl}
\newcommand\Colorhref[3][orange]{\href{#2}{\small\color{#1}#3}}


% \newcommand{\cvreference}[7]{%
%     \textbf{#1}\newline% Name
%     \ifthenelse{\equal{#2}{}}{}{\addresssymbol~#2\newline}%
%     \ifthenelse{\equal{#3}{}}{}{#3\newline}%
%     \ifthenelse{\equal{#4}{}}{}{#4\newline}%
%     \ifthenelse{\equal{#5}{}}{}{#5\newline}%
%     \ifthenelse{\equal{#6}{}}{}{\emailsymbol~\texttt{#6}\newline}%
%     \ifthenelse{\equal{#7}{}}{}{\phonesymbol~#7}}

\begin{document}

\makecvtitle % Print the CV title




%----------------------------------------------------------------------------------------
%	EDUCATION SECTION
%----------------------------------------------------------------------------------------

\section{教育经历}

\cventry{2016--2023}{博士, 软件工程}{上海交通大学}{上海}{中国}
{计算机视觉, 大规模图像检索, 行人重识别, 车辆重识别, 图像生成}  % Arguments not required can be left empty

% \cventry{2013--2015 :}{Master of Engineering, Information Technology}{Indian Institute of Engineering Science \& Technology}{Shibpur(\textit{Formerly} Bengal Engineering and Science University, Shibpur)}{}{}
% %{Advanced exposure to various areas of computer science along with a one and half year research project on Reversible Logic Synthesis.}
% %\cvitem{CGPA :}{7.96/10}
\cventry{2012--2016 }{学士, 软件工程}{西北工业大学}{陕西}{中国}
{}

%----------------------------------------------------------------------------------------
%	PUBLICATION SECTION
%----------------------------------------------------------------------------------------

\section{学术论文发表}
\subsection{期刊论文}
\cventry{2019}{\textbf{Pratik Dutta}, Sriparna Saha, Sanket Pai and Aviral Kumar}{}{Protein-protein Interaction based Generative Model for Improving Gene Clustering}{In \textit{\textbf{Scientific Reports-Nature}} (\textbf{Impact Factor: 4.12})}{}

\subsection{Journal Articles}
\newbibliography{journal}
\bibliographystyle{journal}{plainyrrev}
\nocite{journal}{*}
\bibliography{journal}{journal}
{\large \textsc{Refereed Journal Articles}}



\newbibliography{conference}
\nocite{conference}{*}
\bibliographystyle{conference}{chronological}
\bibliography{conference}{conference}
{\large \textsc{Refereed Conference Publications}}


%----------------------------------------------------------------------------------------
%	WORK EXPERIENCE SECTION
%----------------------------------------------------------------------------------------

\section{科研项目}
\subsection{上海交通大学, 上海}
\cventry{2018, 六月 -- 2019, 二月}{\textit{跨模态行人重识别研究}}{}{}{}
{
  使用双流卷积神经网络进行提取RGB 图片和Infrared 图片的特征。
  设计了一个基于自编码器的重构编码来对齐两个模态的特征。
  基于贝叶斯学习设计了模态和外表恒定的损失函数来进行度量学习,使得学到对于模态和外
  观不变的特征。
  在两个标准数据集上取得了state-of-the-art 性能表现(论文发表在国际会议ICMR 2020)。
}

\cventry{2019, 十月 -- 2020, 八月}{\textit{高效的大规模车辆重识别}}{}{}{}
{使用卷积神经网络CNN 进行连续的特征提取。
设计了一个新型的二进制哈希码学习模块。
提出学习适合于分类的二进制编码,并且考虑到无法直接在二进制码上使用随机梯度下降优
化网络,我们提出了一个交叉优化的优化手段
提出使用Cyclic coordient descent 学习二进制哈希码,以及使用三元损失函数,以及量化损失
函数优化神经网络。
此工作属于第一个使用深度哈希解决大规模的车辆重识别的工作,论文发表在(IEEE TITS JCR
1 区)。
}


\cventry{2021, 二月 -- 2021, 十月}{\textit{基于视觉Transformer 的大规模图像检索}}{}{}{}
{提出一个孪生视觉Transformer 模型用于提取特征。
采取一个局部全局的双流Transformer 模块用于提取细粒度特征。
使用全连接和Tanh 激活函数生成哈希向量。
使用贝叶斯学习来进行保距的度量学习。在三个标准数据集上进行实验,和对比的基准方法
比较可以显著提高模型的性能。
此工作是第一个完全基于transformer 进行深度哈希学习来做图像检索的工作(论文发表在国
际会议ICMR 2022)。
}

\cventry{2022,三月 -- 2022,十月}{\textit{基于视觉Transformer 的深度乘积量化}}{}{}{}
{
  提出了第一个完全基于视觉Transformer 的乘积量化网络, 使用softmax 规避乘积量化中的不
可导问题。
采用一个双支视觉Transformer 来提取更加细粒度的特征,同时尽量减少计算开销。
提出第一个基于直接优化训练目标mAP 的量化损失函数。
模型在大规模基准数据集上取得了显著的性能提升。
}












%----------------------------------------------------------------------------------------
%	Fellowships \& Awards
%----------------------------------------------------------------------------------------




%----------------------------------------------------------------------------------------
%	Academic achievements
%----------------------------------------------------------------------------------------

\section{学术服务}


\cvitem{2021}{\textbf{R\textit{IEEE International Conference on Multimedia and Expo  审稿人} (ICME 2021)}, 深圳, 中国.}

\cvitem{2022}{\textbf{\textit{IEEE International Conference on Multimedia and Expo 审稿人} (ICME 2022)},  台北, 台湾.}

\cvitem{2023}{\textbf{\textit{IEEE International Conference on Acoustics, Speech and Signal Processing 审稿人} (ICASSP 2023)},  罗德岛, 希腊.}

\cvitem{2023}{\textbf{\textit{IEEE International Conference on Multimedia and Expo 审稿人} (ICME 2023)},  布里斯班, 澳大利亚.}

\cvitem{2023}{\textbf{\textit{IEEE Transactions on Multimedia 审稿人} (TMM 2023)}.}




%----------------------------------------------------------------------------------------
%	COMPUTER SKILLS SECTION
%----------------------------------------------------------------------------------------

\section{计算机技能}

\cvitem{编程语言}{Python, PyTorch, Numpy, Tensorflow, C, Java}



%----------------------------------------------------------------------------------------
%	Position of Responsibility SECTION
%----------------------------------------------------------------------------------------



% \section{Referees}


% \begin{tabular}{lr}
% % Referee 1
% \begin{minipage}[t]{3in}
% \textbf{Dr. XXXXX XXXXX}\\
% \textit{Associate Professor, Department of} \\
% \textit{Computer Science \& Engineering}\\
% Institute name\\
% \Letter\ \href{mailto:abc@gmail.com}{abc@gmail.com}
% \end{minipage}
% &
% % Referee 2
% \begin{minipage}[t]{3in}
% \textbf{Dr. XXXXX XXXXX}\\
% \textit{Associate Professor, Department of} \\
% \textit{Computer Science \& Engineering}\\
% Institute name\\
% \Telefon\ +(601) 877-6236\\
% \Letter\ \href{mailto:abc@gmail.com}{abc@gmail.com}
% \end{minipage}
% \\
% \\ % Additional newline for spacing.
% % Referee 3
% \begin{minipage}[t]{3in}
% \textbf{Dr. XXXXX XXXXX}\\
% \textit{Associate Professor, Department of} \\
% \textit{Computer Science \& Engineering}\\
% Institute name\\
% \Telefon\ +(601) 877-6236\\
% \Letter\ \href{mailto:abc@gmail.com}{abc@gmail.com}
% \end{minipage}
&
% \\
% \end{tabular}


\end{document}